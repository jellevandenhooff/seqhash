\section{SeqHash}
\label{s:background}

The SeqHash tree~\cite{versum:ccs14} is a derandomized history-independent
balanced binary tree. %The main design difference between SeqHash and the treap
%is that SeqHash uses it source of randomness over and over again where the
%treap associates only a single priority with each node. 
The shape of a SeqHash is defined by construction. At a high level, SeqHash
builds a tree level by level, and follows a merging process at each level
to determine which nodes in each level should be combined into a new node.
This section describes the construction in detail.

The first level of a SeqHash consists of leaf nodes with all the values in the
SeqHash in their original order.  Those leafs are merged to form the second
level of the SeqHash, and those nodes get merged again to form the third level,
and so forth. This process continues until all nodes have been merged and a
single root node remains on the final level.

%Not all nodes get merged each level; unmerged nodes pass through
%to the next level and might get merged there. The merging process continues
%until all nodes have been merged, and the tree ends with a single root on the
%last level.

To determine which nodes to merge on a single level, SeqHash follows merging
process based on nodes generating pseudorandom bits. Each node's hash is used
to seed a generator, creating a stream of bits coming out of each node.  To
merge nodes, SeqHash performs a bit-by-bit process: First, we sample a new bit
from each node's generator, so that each node has either a 0-bit or a 1-bit
attached. Then, we consider adjacent pairs of nodes.  Whenever the node on the
left generates a 1-bit, and the node on the right generates a 0-bit, we merge
the two nodes and create a new node with the two existing nodes as children.
This new node takes their place in the next level. Requiring a 1-bit on one
side and a 0-bit on the other side ensures that the merging is well defined,
and that there is only one possible interpretation of the merging rule.

After merging two nodes, the merged nodes are no longer considered for the
remainder of the level. The merging continues until merging is no longer possible.
Some nodes might remain unmerged, but can also not merge anymore because both
of their neighbors have been merged with their other neighbors. Such individual
nodes pass through and become part of the next level, and have another chance
to merge on that level.

Because a level ends only once no pairs of adjacent nodes remain unmerged,
at least $2/3$ of the nodes on each level must merge. That means that
the total number of nodes decreases by $1/3$ each level, and so the total
height of the tree is bound by $O(\log n)$.

Intuitively, each level will only consume a limited number of bits. Consider
any pairs of adjacent nodes. For every bit generated, that pair of nodes will
merge with probability $1/4$ (if they have not already merged yet), and so the
expected number of bits generated is 4 and with high probability the nodes
merge after $O(1)$ bits. By applying the union bound, each level ends after
$O(n)$ bits with high probability.

Like the treap, SeqHash's shape is defined locally: Two adjacent nodes that
generate a 1-bit and a 0-bit for their first bit will merge no matter what the
other nodes generate. This means that we can \emph{partially evaluate} the
SeqHash construction for a subsequence, and later on combine the partially
evaluated SeqHash with another partially evaluated SeqHash to construct a
SeqHash for the combined sequence. To compute this partial evaluation, the
construction process is expanded to consider two potential unknown values at
the beginning and end of the input sequence. Whenever a node might merge with
those unknown values, that node itself is marked as unknown as well.  The final
output of a partial evaluation is the set of unknown nodes along with the
single root node on the final level.

\section{A SeqHash-based associative hash function}
\label{s:seqhash}

To build an associative hash function on top of SeqHash, we need a fringe.
SeqHash's partial evaluation is the perfect basis for a fringe. All nodes that
are not unknown nodes will never merge with another node, and so we need not
store them for use in the fringe. Instead, we store only the unknown nodes, and
the root node of the SeqHash.

Examples.

Analysis for random inputs.
