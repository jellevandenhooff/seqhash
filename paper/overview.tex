\section{From history-independent tree to associative hash function}
\label{s:overview}

\begin{figure}[htb]
\begin{center}
\begin{tikzcd}[row sep=small, column sep=large]
  a \arrow[dd, bend right=50, "\hash"'] \arrow[d] \arrow[r, phantom, "||"] & b \arrow[r] \arrow[d] & \arrow[d] a||b \\
  T_a \arrow[d] \arrow[r, phantom, "\text{merge}"] & T_b \arrow[r] \arrow[d] & \arrow[d] T_{a||b} \\
  F_a \arrow[r, phantom, "\cat"] & F_b \arrow[r] & F_{a||b}
\end{tikzcd}
\end{center}
\caption{A diagram showing the relationship between $\hash$, $\cat$, $F$, and $T$. The sequence $s$ has corresponding tree $T_s$ with fringe $F_s$. We can compute $F_s$ using $\hash(s) = F_s$. Two sequences $a$ and $b$ can be concatenated to obtain $a||b$. Using $\cat$, we can compute $F_{a||b}$ using $\cat(F_a, F_b) = F_{a||b}$.}
\label{fig:diagram}
\end{figure}

The problem we focus on in this paper is on how to compute a hash (or
fingerprint) over arbitrary input sequences, supporting efficient merging of
hashes. Formally, our inputs are sequences $\Sigma^*$ consisting of symbols
from the alphabet $\Sigma = \{0,1\}^k$. 

Our basic building block will be an efficient implementation of a
history-independent balanced binary tree that takes an input $s$ and builds a
balanced tree $T_s$ of height $O(\log |s|)$. This tree must support efficient
merging, so that we can merge trees $T_a$ and $T_b$ into $T_{a||b}$
corresponding to the concatenation $a||b$ in time $O(\log |a| + \log |b|)$.

Using the tree as a basic building block, we can compute a Merkle-style hash of
the tree by recursively hashing nodes. The hash of each node is a cryptographic
hash of the hashes of its children, and the hash of the tree is the hash of the
root node.

This hash itself is not suitable as an associative hash. While we can easily
distribute the small hash of the root of $T_s$, we cannot easily pass around
$T_s$ itself since the tree contains $O(|s|)$ leafs.  Yet we need to know the
structure of $T_a$ and $T_b$ to compute $T_{a||b}$, and that is lost when
looking at just the hashes of the roots of the trees.

Instead, our construction relies on the existence of a small representation
$F_s$ of the \emph{fringe} of a tree $T_s$ such that we can still compute the
concatenation $F_{a||b}$ of $F_a$ and $F_b$ without access to either $T_a$ or
$T_b$. We show that such a fringe exists for both the treap in \S\ref{s:treap}
and for SeqHash in \S\ref{s:seqhash}.

Our final construction in both cases consists of two functions, $\hash$ and
$\cat$. The function $\hash$ computes the fringe $F_s$ from $s$, and $\cat$
computes $F_{a||b}$ from $F_a$ and $F_b$ as shown in~Figure~\ref{fig:diagram}.
The fringe $F_s$ has size $O(\log |s|)$ and can be computed in time $O(n)$. The
$\cat$ function has runtime $O(\log |a| + \log |b|)$.

%As an external building block, SeqHash rely on an external cryptographic
%pseudorandom function $\prf$ that maps inputs in $\{0,1\}^*$ to an infinite
%sequence of random output bits. This source of randomness is used to determine
%the shape of the tree no matter what the input looks like. For a hash function to be widely useful as a summary function, anyone must be
%able to compute it -- and consequently, we cannot assume that our choice of pseudorandom
%function remains secret (as in a PRF). Instead, SeqHash

%hide the key to the pseudorandom function to hide the behavior of our tree from
%an adversary.

%XXX An argument as to why associative hash functions need randomization.
