\section{A treap-based associative hash function}
\label{s:treap}

\paragraph{Why a randomized data-structure?}

Our core design conists of three steps: First, a sequence $s$ is converted in a
tree $T_s$. This tree must be the same no matter how it is constructed; hashing
it from $s$, or by hashing the beginning and end of $s$, and concatenating the
trees. As a strawman, this section will consider a treap-based hash function.

To build a treap over a sequence $s = a_1 a_2 \cdots a_n$, we associate a
random priority $\prf(a_i)$ with every node. The node with highest priority
will be placed at the top of tree, breaking ties by index in $s$. This treap
always has the same shape for the same sequence as long as the same $\prf$ is
used. 

To calculate the hash of a treap, we recursively calculate a hash of the tree
nodes by concatenating the hashes of the previous elements, and feeding those
into a cryptographic hash function. For this hash to be meaningful, it must be
the same for all ways of constructing the tree. This is the history independence
property that a treap gives us by relying on priorities.

For some use-cases, the treap is a balanced tree -- namely, if $\prf$ behaves
randomly.  That is the case of the alphabet $\Sigma = \{0,1\}^k$ is large and
no two elements in the sequence have the same value. In other scenarios, the
treap might be very unbalanced.  For example, if $a_1 = a_2 = \ldots = a_n$ the
treap will degenerate into a tree of depth $n$, or a linked-list. This is the
problem that we will address with SeqHash. For the rest of this section we will
assume that the treap has a balanced shape.

To concatenate the hash of two sequences $a$ and $b$, we merge their
corresponding trees. The treap merge algorithm is simple: the root node
of the merged tree is the root node is the root node of $a$ or $b$ with higher
priority, and the rest of the tree is merged recursively:

\begin{Verbatim}[commandchars=\\\{\},numbers=left,firstnumber=1,stepnumber=1,codes={\catcode`\$=3\catcode`\^=7\catcode`\_=8}]
\PY{k}{def} \PY{n+nf}{treap\PYZus{}merge}\PY{p}{(}\PY{n}{a}\PY{p}{,} \PY{n}{b}\PY{p}{)}\PY{p}{:}
    \PY{k}{if} \PY{n}{a} \PY{o}{==} \PY{n+nb+bp}{None} \PY{o+ow}{or} \PY{n}{b} \PY{o}{==} \PY{n+nb+bp}{None}\PY{p}{:}
        \PY{k}{return} \PY{n}{a} \PY{o+ow}{or} \PY{n}{b}
    \PY{k}{if} \PY{n}{a}\PY{o}{.}\PY{n}{priority} \PY{o}{\PYZgt{}} \PY{n}{b}\PY{o}{.}\PY{n}{priority}\PY{p}{:}
        \PY{k}{return} \PY{n}{a}\PY{o}{.}\PY{n}{update}\PY{p}{(}\PY{n}{right}\PY{o}{=}\PY{n}{treap\PYZus{}merge}\PY{p}{(}\PY{n}{a}\PY{o}{.}\PY{n}{right}\PY{p}{,} \PY{n}{b}\PY{p}{)}\PY{p}{)}
    \PY{k}{else}\PY{p}{:}
        \PY{k}{return} \PY{n}{b}\PY{o}{.}\PY{n}{update}\PY{p}{(}\PY{n}{left}\PY{o}{=}\PY{n}{treap\PYZus{}merge}\PY{p}{(}\PY{n}{a}\PY{p}{,} \PY{n}{b}\PY{o}{.}\PY{n}{left}\PY{p}{)}\PY{p}{)}
\end{Verbatim}


This treap merge routine is efficient as it only descends into one node. If
the treaps are balanced, merging two treaps take time $O(\log |a| + \log |b|)$.
More importantly, merging two treaps only requires access to a small part of
the original trees. 

When merging two trees, we only ever access the right child of $a$, and only
ever access the left child of $b$. To support any kind of merging, we only need
to store the nodes on the outside of the tree: Nodes that can be reached either
by only going left, or by only going right from the root of the tree. We call
these nodes the \emph{fringe} of the tree.  The size of the fringe is
proportional is at most twice the height of the tree, or $O(\log |s|)$ for
sequences $s$ with balanced trees.

\begin{Verbatim}[commandchars=\\\{\},numbers=left,firstnumber=1,stepnumber=1,codes={\catcode`\$=3\catcode`\^=7\catcode`\_=8}]
\PY{k}{def} \PY{n+nf}{extract\PYZus{}fringe}\PY{p}{(}\PY{n}{node}\PY{p}{,} \PY{n}{side}\PY{p}{)}\PY{p}{:}
    \PY{k}{if} \PY{n}{node} \PY{o}{==} \PY{n+nb+bp}{None}\PY{p}{:}
        \PY{k}{return} \PY{n+nb+bp}{None}
    \PY{k}{if} \PY{n}{side} \PY{o}{==} \PY{l+s}{\PYZsq{}}\PY{l+s}{left}\PY{l+s}{\PYZsq{}}\PY{p}{:}
        \PY{k}{return} \PY{n}{node}\PY{o}{.}\PY{n}{update}\PY{p}{(}\PY{n}{left}\PY{o}{=}\PY{n}{extract\PYZus{}fringe}\PY{p}{(}\PY{n}{node}\PY{o}{.}\PY{n}{left}\PY{p}{,} \PY{l+s}{\PYZsq{}}\PY{l+s}{left}\PY{l+s}{\PYZsq{}}\PY{p}{)}\PY{p}{,} 
                           \PY{n}{right}\PY{o}{=}\PY{n+nb}{hash}\PY{p}{(}\PY{n}{node}\PY{o}{.}\PY{n}{right}\PY{p}{)}\PY{p}{)}\PY{o}{.}
    \PY{k}{elif} \PY{n}{side} \PY{o}{==} \PY{l+s}{\PYZsq{}}\PY{l+s}{right}\PY{l+s}{\PYZsq{}}\PY{p}{:}
        \PY{k}{return} \PY{n}{node}\PY{o}{.}\PY{n}{update}\PY{p}{(}\PY{n}{left}\PY{o}{=}\PY{n+nb}{hash}\PY{p}{(}\PY{n}{node}\PY{o}{.}\PY{n}{left}\PY{p}{)}\PY{p}{,}
                           \PY{n}{right}\PY{o}{=}\PY{n}{extract\PYZus{}fringe}\PY{p}{(}\PY{n}{node}\PY{o}{.}\PY{n}{right}\PY{p}{,} \PY{l+s}{\PYZsq{}}\PY{l+s}{right}\PY{l+s}{\PYZsq{}}\PY{p}{)}\PY{p}{)}
    \PY{k}{else}\PY{p}{:} \PY{c}{\PYZsh{} root}
        \PY{k}{return} \PY{n}{node}\PY{o}{.}\PY{n}{update}\PY{p}{(}\PY{n}{left}\PY{o}{=}\PY{n}{extract\PYZus{}fringe}\PY{p}{(}\PY{n}{node}\PY{o}{.}\PY{n}{left}\PY{p}{,} \PY{l+s}{\PYZsq{}}\PY{l+s}{left}\PY{l+s}{\PYZsq{}}\PY{p}{)}\PY{p}{,}
                           \PY{n}{right}\PY{o}{=}\PY{n}{extract\PYZus{}fringe}\PY{p}{(}\PY{n}{node}\PY{o}{.}\PY{n}{right}\PY{p}{,} \PY{l+s}{\PYZsq{}}\PY{l+s}{right}\PY{l+s}{\PYZsq{}}\PY{p}{)}\PY{p}{)}
\end{Verbatim}


After merging, most nodes in the tree remain the same, and so their hashes
remain the same. This means we can re-compute the hashes of the changed nodes
in time $O(\log |a| + \log |b|)$, because the hashes of all the other nodes
haven't changed.
